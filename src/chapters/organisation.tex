% see B6.2a
\chapter{Projektaufbauorganisation}

% Diesen Abschnitt anpassen.
Das Projekt \placeholder\ wird in der Abteilung \enquote{\varCompanyDepartment} umgesetzt. Projektleiter ist \placeholder\ und für die IPA auch die verantwortliche Fachkraft. Für die fachliche Umsetzung des hier beschriebenen Projekts ist ausschliesslich die Zusammenarbeit zwischen \placeholder\ und dem Kandidaten notwendig.

Im Unterschied zur üblichen Arbeit im Betrieb, werden zwei Experten die Arbeit des Kandidaten begleiten und bewerten. Die Projektaufbauorganisation ist in \ref{fig:organigram} visualisiert.

\begin{figure}[H]
  \begin{multicols}{2}
    \begin{forest}
      for tree={draw,grow'=0,folder,align=left}
      [\textbf{\varCompany}
        [\textbf{HR}
          [(BB) \\ \varVocationalTrainer]
        ]
        [...]
        [\textbf{\varCompanyDepartment}
          [(VF) \\ \varResponsibleSpecialist]
          [(K) \\ \varCandidate]
        ]
      ]
    \end{forest}

    \begin{forest}
      for tree={draw,grow'=0,folder,align=left}
      [\textbf{\varExaminationBoard}
        [\textbf{\varExaminationBoardDepartment}
          [(HEX) \\ \varPrimaryExpert]
          [(NEX) \\ \varSecondaryExpert]
        ]
      ]
    \end{forest}
  \end{multicols}
  \caption[\enquote{Organigramm der am Projekt teilnehmenden Personen} visualisiert mit TikZ Forest]{\gls{Organigramm} der am Projekt teilnehmenden Personen}
  \label{fig:organigram}
\end{figure}